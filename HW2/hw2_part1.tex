\documentclass{article}
\usepackage{graphicx, subfig, fancyhdr, amsmath, amssymb, amsthm, url, hyperref, geometry, listings, xcolor}
\usepackage[utf8]{inputenc}
\usepackage[margin=1in]{geometry}
\usepackage{amsmath}
\usepackage{relsize}

% Fix Unicode character issues
\DeclareUnicodeCharacter{2212}{-}
\DeclareMathOperator{\Tr}{Tr}

\lstset{
    language=Matlab,
    basicstyle=\ttfamily\small,
    numbers=left,
    numberstyle=\tiny,
    stepnumber=1,
    frame=single,
    backgroundcolor=\color{gray!10},
    keywordstyle=\color{blue}\bfseries,
    commentstyle=\color{green!50!black},
    stringstyle=\color{red},
    breaklines=true,
    breakatwhitespace=true,
    showstringspaces=false,
    tabsize=4
}

% Author Information
\newcommand{\FirstAuthor}{Mohammad Parsa Dini -- Student ID: 400101204}
\newcommand{\exerciseset}{HW 2 (Solution) Part I}

% Page Formatting
\fancypagestyle{plain}{}
\pagestyle{fancy}
\fancyhf{}
\fancyhead[RO,LE]{\sffamily\bfseries\large Sharif University of Technology}
\fancyhead[LO,RE]{\sffamily\bfseries\large EE 25-645: Neuroscience}
\fancyfoot[LO,RE]{\sffamily\bfseries\large HW 2 Solution}
\fancyfoot[RO,LE]{\sffamily\bfseries\thepage}
\renewcommand{\headrulewidth}{1pt}
\renewcommand{\footrulewidth}{1pt}

% Custom Commands
\newcommand{\circledtimes}{\mathbin{\text{\large$\bigcirc$}\kern-0.9em\times}}

% Image Path
\graphicspath{{figures/}}

%-------------------------------- Title ----------------------------------
\title{ 
\includegraphics[width=3cm]{logo.png} \\ 
% Adjust width as needed 
Neuroscience of Learning, Memory, Cognition \par \exerciseset } 
\author{\FirstAuthor }
\date{}
%--------------------------------- Document ----------------------------------
\begin{document}
\maketitle

% ---------------------------------------------------------------------------------------------%
\section*{1. Linear Filter Model of a Synapse }

We assume that $\rho_b (t) = \sum_i \delta(t-t_i)$ is the spike-train signal,  $K(t) = e^{-\tau/\tau_s}$ is the exponential decay signals and $P_s(t) = \sum_{t>t_i} K(t-t_i)$ is the probability of that postsynaptic ion channels are open at time $t$. 

\subsection*{1.1 Background and Introduction}
Given the spike train times $\mathcal{T}=\{1 ms, 3 ms, 7 ms, 9 ms, 18 ms\}$, we can model as the firing-rate as \begin{equation*}
    \rho_b(t) = \sum_{\tau \in \mathcal{T}} \delta(t-\tau) = \delta(t-0.001) + \delta(t-0.003) + \delta(t-0.007) + \delta(t-0.009) + \delta(t-0.018) 
\end{equation*}

Now, given $\tau_s = 5ms$, we can compute $P_S(10^{-2})$ as:
\begin{equation*}
    P_s(10ms) = K(1ms)+ K(3ms) + K(7ms) + K(9ms) = 
    e^{-\frac{1}{5}} + e^{-\frac{3}{5}} + e^{-\frac{5}{5}} + e^{-\frac{9}{5}} \approx 1.7794
\end{equation*}

We can easily see that \textbf{higher $\tau_s$} means slower decay of postsynaptic response and therefore, longer-lasting influence of each spike, temporal summation is more significant. However,\textbf{ lower $\tau_s$} means faster decay and thus each spike’s effect is brief, less summation.
% --------------------------------------------      --------------------------------------------%
\subsection*{1.2 Shifting property and convolution interpretation}
Assuming that the spiking train $\rho_b(t) = \delta(t-1) + \delta(t-2) + \delta(t-3)$ passes through the kernel $h(t)=e^{-t} u(t)$ , the filtered output will be:
\begin{equation*}
    y(t) = h(t) * \rho_b(t) = h(t-1) + h(t-2) + h(t-3) = e^{-t+1} u(t-1) + e^{-t+2} u(t-2) + e^{-t+3} u(t-3) 
\end{equation*}
% --------------------------------------------      --------------------------------------------%
\subsection*{1.3 Calculating the system’s response to a spike train}
We have the total synaptic input current as $I_s(t) = \sum_{b=1}^n w_b \int_{-\infty}^{t} K(t-\tau) \rho_b(t)$, since  $\rho_b(t) = \sum_{i} \delta(t-t_i^{(b)})$ 
, we obtain that: 
\begin{equation*}
    I_s(t) = \mathlarger{\sum}_{b=1}^n w_b \mathlarger{\sum}_{t_i^{(b)}<t} K(t - t_i^{(b)}) 
\end{equation*}
where $K(t) = e^{-t/\taus}$ and $\taus=5ms$. We have two synapses synapse1 and synapse2 who spike at times $\mathcal{T}_1 = \{1ms, 6ms \}$ and $\mathcal{T}_2 = \{2ms, 8ms , 11ms\}$ with synaptic weights $w_1= 1$ and $w_2=0.5$, respectively.\\
Now we will compute total input current at $t=15ms$. Let $I_1, I_2$ be the Synapse 1 and 2 contributions at $t=15ms$: 
\begin{align*}
    I_1 &= w_1 (K(14\text{ms}) + K(9\text{ms})) = 1 \times (e^{-1.8} + e^{-2.8}) = 0.2261095 \\
    I_2 &= w_2 (K(13\text{ms}) + K(7\text{ms}) + K(4\text{ms})) = 0.2 \times (e^{-2.6} + e^{-1.4} + e^{-0.8}) = 0.3850998 \\
    I_s &= I_1 + I_2 = 0.2261095 + 0.3850998 = 0.6112093 \approx 0.611
\end{align*}

Here is the code of the simulation of the spikes:
\begin{lstlisting}
% --------------- Simulation of Spikes -----------------
tau_s = 5.0;
t_vals = linspace(0, 25, 1000);

spikes1 = [1, 6];
spikes2 = [2, 8, 11];

% Define the P_b function
function result = P_b(t, spikes, tau_s)
    result = zeros(size(t));
    for i = 1:length(spikes)
        spike_time = spikes(i);
        result = result + (t > spike_time) .* exp(-(t - spike_time) / tau_s);
    end
end

% Calculate postsynaptic responses
P1 = P_b(t_vals, spikes1, tau_s);
P2 = P_b(t_vals, spikes2, tau_s);
Is = P1 + 0.5 * P2;
\end{lstlisting}

In the next page, you can see the result of the simulation where we can clearly see that after each neuron spikes, the $I_s(t)$ jumps a little bit with respect to the weights and then it would continue its decay. Here the graphs of the current $I_s(t)$ and post-synaptic responses are depicted:
\begin{figure}[h!]
    \centering
    \includegraphics[scale=0.33]{img1.png}
    \caption{total synaptic input current $I_s(t)$ over time.}
    \label{fig:V}
\end{figure}



\end{document}
