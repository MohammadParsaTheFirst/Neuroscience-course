\documentclass{article}
\usepackage{graphicx, subfig, fancyhdr, amsmath, amssymb, amsthm, url, hyperref, geometry, listings, xcolor}
\usepackage[utf8]{inputenc}
\usepackage[margin=1in]{geometry}

% Fix Unicode character issues
\DeclareUnicodeCharacter{2212}{-}
\DeclareMathOperator{\Tr}{Tr}

\lstset{
    language=Matlab,
    basicstyle=\ttfamily\small,
    numbers=left,
    numberstyle=\tiny,
    stepnumber=1,
    frame=single,
    backgroundcolor=\color{gray!10},
    keywordstyle=\color{blue}\bfseries,
    commentstyle=\color{green!50!black},
    stringstyle=\color{red},
    breaklines=true,
    breakatwhitespace=true,
    showstringspaces=false,
    tabsize=4
}

% Author Information
\newcommand{\FirstAuthor}{Mohammad Parsa Dini -- Student ID: 400101204}
\newcommand{\exerciseset}{HW 1 (Solution)}

% Page Formatting
\fancypagestyle{plain}{}
\pagestyle{fancy}
\fancyhf{}
\fancyhead[RO,LE]{\sffamily\bfseries\large Sharif University of Technology}
\fancyhead[LO,RE]{\sffamily\bfseries\large EE 25-645: Neuroscience}
\fancyfoot[LO,RE]{\sffamily\bfseries\large HW 1 Solution}
\fancyfoot[RO,LE]{\sffamily\bfseries\thepage}
\renewcommand{\headrulewidth}{1pt}
\renewcommand{\footrulewidth}{1pt}

% Custom Commands
\newcommand{\circledtimes}{\mathbin{\text{\large$\bigcirc$}\kern-0.9em\times}}

% Image Path
\graphicspath{{figures/}}

%-------------------------------- Title ----------------------------------
\title{ 
\includegraphics[width=3cm]{logo.png} \\ 
% Adjust width as needed 
Neuroscience of Learning, Memory, Cognition \par \exerciseset } 
\author{\FirstAuthor }
\date{}
%--------------------------------- Document ----------------------------------
\begin{document}
\maketitle

% --------------------------------------------------
\section*{2. Simple Dynamic Model}

Dr.~Eugene Izhikevich proposed the following spiking neuron model:

\begin{align}
\frac{dV}{dt} &= -\alpha V + \beta V^{2} + \gamma - u + I(t), \\
\frac{du}{dt} &= a(-u + bV),
\end{align}

where $V$ is the membrane potential and $u$ is a recovery variable representing combined ionic currents.

The model includes the following threshold and reset condition:


\[
\text{if } V \geq 30 \,\text{mV} \quad \Rightarrow \quad 
\begin{cases}
V \leftarrow c, \\
u \leftarrow u + d,
\end{cases}
\]


meaning that when the membrane potential reaches $30$ mV, the potential is reset to $c$ and the recovery variable $u$ is incremented by $d$.

\vspace{1em}
To simulate the dynamic model numerically, we discretize the system. At each timestep we update the variables by adding the derivative multiplied by the timestep size:


\[
x(t+\Delta t) = x(t) + \frac{dx(t)}{dt}\,\Delta t.
\]



Here is the MATLAB code for simulating this model:

\begin{lstlisting}
% --------------- Izhikevich Neuron Model -----------------
% Simulation settings
dt = 0.1;       % time step (ms)
T  = 1000;      % total time (ms)
steps = T/dt;

% Initialize variables
V = -65;        % membrane potential
u = b*V;        % recovery variable
Vs = zeros(1,steps);
us = zeros(1,steps);
ts = (0:steps-1)*dt;

% Dynamics functions
dVdt = @(V,u) -alpha*V + beta*V.^2 + gamma - u + I;
dudt = @(V,u) a*(-u + b*V);

% Simulation loop
for i = 1:steps
    V = V + dt*dVdt(V,u);
    u = u + dt*dudt(V,u);
    % Threshold condition
    if V >= 30
        V = c;
        u = u + d;
    end
    Vs(i) = V;
    us(i) = u;
end
\end{lstlisting}

\subsection*{2.1 Regular Spiking}

With parameters $a=0.02,\; b=0.2,\; c=-65,\; d=8,\; I=10$ and assuming $t \to \infty$ (steady state), we set:


\[
\frac{du}{dt} = \frac{dV}{dt} = 0.
\]



From $\frac{du}{dt}=0$:


\[
-u + bV = 0 \quad \Rightarrow \quad u = bV = 0.2V.
\]



Substituting into $\frac{dV}{dt}=0$:


\[
-(-5)V + 0.04V^2 + 150 - u + 10 = 0,
\]




\[
0.04V^2 + 4.8V + 160 = 0.
\]



The discriminant is negative:


\[
\Delta = (4.8)^2 - 4(0.04)(160) = -2.56 < 0,
\]


so the equilibrium points are complex and no real fixed point exists. This explains why the neuron exhibits repetitive spiking under constant input.


Here is the result of simulation, showing the recovery variable $u(t)$:
\begin{figure}[h!]
    \centering
    \includegraphics[scale=0.5]{au.png}
    \caption{Recovery variable $u(t)$ over time.}
    \label{fig:u}
\end{figure}

Here is the result of simulation, showing the membrane potential $V(t)$:
\begin{figure}[h!]
    \centering
    \includegraphics[scale=0.5]{av.png}
    \caption{Membrane potential $V(t)$ over time.}
    \label{fig:V}
\end{figure}
%-----------------------------------------------------

\subsection*{2.2 Fast spiking}

With parameters $a=0.1,\; b=0.2,\; c=-65,\; d=2,\; I=10$ and assuming $t \to \infty$ (steady state), we set: 

\[
\frac{du}{dt} = \frac{dV}{dt} = 0.
\]

From $\displaystyle\frac{du}{dt} = 0$:

$$u = bV = 0.2V$$

Substitute into $\displaystyle\frac{dV}{dt} = 0$:

$$0.04V^2 + 5V + 150 - 0.2V + 10 = 0$$

$$\Rightarrow 0.04V^2 + 4.8V + 160 = 0$$

Discriminant:

$$\Delta = (4.8)^2 - 4(0.04)(160) = 23.04 - 25.6 = -2.56 < 0$$

\textbf{Conclusion:} No real fixed point exists. The system is driven into oscillatory spiking once the threshold/reset is included. This matches the Fast Spiking behavior.\\


Here is the result of simulation, showing the recovery variable $u(t)$:
\begin{figure}[h!]
    \centering
    \includegraphics[scale=0.3]{bu.png}
    \caption{Recovery variable $u(t)$ over time.}
    \label{fig:u}
\end{figure}

Here is the result of simulation, showing the membrane potential $V(t)$:
\begin{figure}[h!]
    \centering
    \includegraphics[scale=0.3]{bv.png}
    \caption{Membrane potential $V(t)$ over time.}
    \label{fig:V}
\end{figure}

% --------------------------------------------------

\subsection*{2.3 Chattering}
We set steady-state conditions:

$$\frac{dV}{dt} = -\alpha V + \beta V^2 + \gamma - u + I = 0, \quad \frac{du}{dt} = a(-u + bV) = 0$$

Parameters: $\alpha = -5,\; \beta = 0.04,\; \gamma = 150,\; a = 0.02,\; b = 0.2,\; I = 10$.

From $\displaystyle\frac{du}{dt} = 0$:

$$u = bV = 0.2V$$

Substitute into $\displaystyle\frac{dV}{dt} = 0$:

$$0.04V^2 + 5V + 150 - 0.2V + 10 = 0$$

$$\Rightarrow 0.04V^2 + 4.8V + 160 = 0$$

Discriminant:

$$\Delta = (4.8)^2 - 4(0.04)(160) = 23.04 - 25.6 = -2.56 < 0$$

\textbf{Conclusion:} No real fixed point exists. With the reset condition, the neuron produces bursts of spikes (chattering), rather than tonic firing.
\\
\nextpage

Here is the result of simulation, showing the recovery variable $u(t)$:
\begin{figure}[h!]
    \centering
    \includegraphics[scale=0.3]{cu.png}
    \caption{Recovery variable $u(t)$ over time.}
    \label{fig:u}
\end{figure}

Here is the result of simulation, showing the membrane potential $V(t)$:
\begin{figure}[h!]
    \centering
    \includegraphics[scale=0.3]{cv.png}
    \caption{Membrane potential $V(t)$ over time.}
    \label{fig:V}
\end{figure}

\end{document}
